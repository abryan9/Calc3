\documentclass[12pt, letterpaper]{article}
\usepackage{graphicx} % graphics package (image insertion and formatting)
\usepackage{amsmath}  % allows access to the equation* environment
\usepackage{verbatim} % comment blocks
\graphicspath{{images/}}

\title{Defaulyt document}
\author{Alex Bryan}
\date{January 2024}

\begin{document}
\maketitle

\begin{comment}

\textbf{}                                        :for bold
\underline{}                                     :to underline
\textit{}                                        :for italics
\emph{}                                          :for context-based emphasis
\verb|text|                                      :like emphasis in terms of auto-formatting
\includegraphics[options]{name}                  :includes graphic
\\                                               :newline, can also use double enter to start new paragraph
\newline


\begin{figure}[hpt]                              :[x] denotes ALLOWED page placement. h=source (with text), p=float (on own page w/o text), t=top of page, b=bottom of page, !=override formatting
    \centering
    \includegraphics[width=0.75\textwidth]{name} :configure figure to be scaled to 75% width
    \caption[short]{title}                       :title configurations for the figure
    \label{fig:name}                         :labelling the figure with fig:name allows for \ref access. A figure number is automatically configured
\end{figure}

\begin{table}[h!]
    \centering
    \begin{tabular}{||c|c|c||}                   :create table. "c c c" denotes 3 centered rows, r and l can be used for right and left allignment as well. "|" indicates vertical column lines
                                                 :  "||" indicates double vertical column lines
        \hline                                   :horizontal line
        Col1 & Col2 & Col3 \\ [0.5ex]            :"&" demarcates individual cells, \\ indicates when to enter next row, [(X)ex] adds space below of X width
        \hline\hline                             :double \hline will split the vertical line within the table, creating a "floating" section
        cell1 & cell2 & cell3 \\
        cell4 & cell5 & cell6 \\  
        cell7 & cell8 & cell9 \\ [1ex]
        \hline
    \end{tabular}
    \caption{Caption}
    \label{table:name}
\end{table}                                      :NOTE: can use TablesGenerator.com to generate tables and export as .tex code

\ref{fig/table:name}                             :reference a labelled figure/table
\pageref{fig/table:name}                         :reference page number of a labelled figure/table


\begin{abstract}                                 :create an abstract. This is inserted after \begin{document}
    Abstract
\end{abstract}

\tableofcontents                                 :add table of contents. Ennumerated chapters/sections/ect. are automatically included
\addcontentsline{toc}{section}{Section Name}     :unnumbered sections can be included with this after \section*{Section Name} declaration. {toc} and {section} can be adapted to include unnumbered sections in other formatting things
\chapter{Chapter name}                           :label a chapter
\section{Section name}                           :label a section, if chapter is used it will format as {Chapter number}.{Section number} "Section name"
\subsection{Subsection name}                     :label a subsection. Chains like chapters do to sections
\section*{Section name}                          :unnumbered section. * in general unnumbers auto-formatted commands


\begin{itemize}                                  :bulleted list
    \item 
\end{itemize}

\begin{enumerate}                                :numbered list
    \item 
\end{enumerate}


$ equation $                                     :inline equations
\( equation \)
\begin{math}
    equation
\end{math}

\[ equation \]                                   :display equations. Appear centered on screen and in their own line. Can be enumerated or listed.
\begin{equation}
    equation
\end{equation}
\begin{displaymath}
    equation
\end{displaymath}
\begin{equation*}
    equation
\end{equation*}

\dots                                            :elipses
X^{n}                                            :superscript
X_{n}                                            :subscript
\int_{min}^{max}                                 :integral (with subscript and superscript for bounds)
\frac{num}{den}                                  :fractions
\omega                                           :lowercase Greek symbols
\delta
\Omega                                           :uppercase Greek symbols
\Delta
\sin(x)                                          :trig functions
\log(x)                                          :log function
    
\end{comment}
\end{document}